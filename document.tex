% \documentclass[a4paper, 11pt, oneside]{book} % A4 paper size and default 11pt font size
\documentclass[a4paper, 14pt, oneside]{book} % A4 paper size and default 14pt font size

\newcommand*{\plogo}{\fbox{$\mathcal{PL}$}} % Generic dummy publisher logo

%\usepackage[utf8]{inputenc} % Required for inputting international characters
%\usepackage[T1]{fontenc} % Output font encoding for international characters
%\usepackage{stix} % Use the STIX fonts

% 导入中文宏
\usepackage{ctex}
\usepackage{float}
\usepackage{indentfirst}
\usepackage{geometry}
\usepackage{amsmath}
\usepackage{setspace}
\usepackage{graphicx}
\numberwithin{equation}{subsection}
\usepackage{cite}
\usepackage{fancyhdr}
\usepackage{color}
\usepackage{appendix}
\usepackage{booktabs}  %table
\usepackage{listings} 
\usepackage{xcolor} 
\usepackage{fontspec}
\setmonofont{Consolas}

\geometry{left=2cm,right=2cm,top=2cm,bottom=2cm}
\setlength{\parindent}{2em}
	\vspace{10mm}

\lstset{
	backgroundcolor=\color{white},   % 选择代码背景,必须加上\ usepackage {color}或\ usepackage {xcolor}.
	basicstyle=\footnotesize,        % 设置代码字号.
	breakatwhitespace=false,         % 设置是否当且仅当在空白处自动中断.
	breaklines=true,                 % 设置自动断行.
	captionpos=b,                    % 设置标题位置.
	commentstyle=\color{green},    % 设置注释格式
	deletekeywords={...},            % 是否删除给定语言的关键词.
	escapeinside={\%*}{*)},          % 是否在代码中添加LaTex.
	extendedchars=true,              % 是否允许使用非ASCII字符; 仅适用于8位编码,不适用于UTF-8. 
	frame=single,	                   % 给代码区添加边框.
	keepspaces=true,                 % 保留空格(useful for keeping indentation of code (possibly needs columns=flexible).
	keywordstyle=\color{blue},       % 关键字显示风格.
	language=Octave,                 % 使用的语言.
	morekeywords={*,...},            % 是否需要添加其他的关键词.
	numbers=left,                    % 给代码添加行号,可取值none, left, right.
	numbersep=5pt,                   % 设置行号与代码之间的间隔
	numberstyle=\tiny\color{gray}, % 行号的字号和颜色
	rulecolor=\color{black},         % 边框颜色,如果没有设置,框架颜色可以在非黑色文本中的换行符上更改(例如 text (e.g. comments (green here)))
	showspaces=false,                % 显示每个地方添加特定下划线的空格; 覆盖了'showtringspaces'
	showstringspaces=false,          % 仅在字符串中允许空格
	showtabs=false,                  % show tabs within strings adding particular underscores
	stepnumber=2,                    % the step between two line-numbers. If it's 1, each line will be numbered
	stringstyle=\color{purple},     % string literal style
	tabsize=2,	                   % 将默认tab设置为2个空格
	title=\lstname                   % show the filename of files included with \lstinputlisting; also try caption instead of title
}
\pagestyle{fancy}
\lhead{}%左页眉
\chead{RISC-V on T-Core} %中间内容
\rhead{}  % 右边内容
\renewcommand\thesection{\arabic{section}.}
\renewcommand\thesubsection{\thesection\arabic{subsection}}
\renewcommand\thesubsubsection{\thesubsection.\arabic{subsubsection}}

\begin{document}
	\newpage
	\begin{titlepage} % Suppresses displaying the page number on the title page and the subsequent page counts as page 1
		
		\raggedleft % Right align the title page	
		\rule{1pt}{\textheight} % Vertical line
		\hspace{0.05\textwidth} % Whitespace between the vertical line and title page text
		\parbox[b]{0.75\textwidth}{ % Paragraph box for holding the title page text, adjust the width to move the title page left or right on the page
			
			{\Huge\bfseries  课程实验报告}\\[2\baselineskip] % Title
			{\LARGE\textit{RISC-V on T-Core}}\\[4\baselineskip] % Subtitle or further description
			{\Large\textit{MaTrixV Team}} % Author name, lower case for consistent small caps
			
			\vspace{0.5\textheight} % Whitespace between the title block and the publisher
			% {\noindent }\\[\baselineskip] % Publisher and logo
		}
	\end{titlepage}

	\newpage
	\tableofcontents
	
	\newpage
	\section{基础篇}
		\subsection{RISC-V简介}
			\subsubsection{RISC-V发展过程}			
				\begin{enumerate}
					\item
						RISC(精简指令集计算机)和CISC(复杂指令集计算机)是当前CPU的两种架构。早些年,市面上只有CISC指令集,后来IBM的研究员通过统计的方法发现,传统CISC处理器中,五分之一的指令承担了五分之四的工作,而剩下五分之四的指令基本没有被使用,或者很少使用,这样,既浪费了CPU的核心面积,增大了功耗,还降低了效率。于是,RISC应运而生。
					\item
						RISC的指令数目较CISC少,CISC中的一些复杂指令,RISC需要用多条简单指令来实现。但指令字等长,效率高,功耗低,并发性高。且内部寄存器丰富,更强调对寄存器的合理调用。但高性能RISC处理器成本高,性价比低,且不同公司的RISC芯片几乎无法通用,生态环境较X86的CISC而言更闭塞,通用性完全无法和X86相比,这就是RISC最大的弊端。
					\item
						20世纪末和21世纪初,市面上绝大多数核心指令集都是不开源的。2010年,加州大学伯克利分校的David A. Patterson教授团队在3个月内开发出完全开源指令集RISC-V,RISC-V指令集是基于精简指令集计算(RISC)原理建立的开放指令集架构(ISA),RISC-V是在指令集不断发展和成熟的基础上建立的全新指令。RISC-V指令集完全开源,设计简单,易于移植Unix系统,模块化设计,完整工具链,同时有大量的开源实现和流片案例,已在社区得到大力支持。
					\item
						它虽然不是第一个开源的的指令集(ISA),但它是第一个被设计成可以根据具体场景可以选择适合的指令集的指令集架构。基于RISC-V指令集架构可以设计服务器CPU、家用电器CPU、工控CPU和传感器中的CPU等。
				\end{enumerate}
				\begin{figure}[!htbp]
					\centering
					\includegraphics[scale=0.8]{img/one.png}
				\end{figure}
			\subsubsection{RISC-V指令结构}
				\begin{enumerate}
					\item 
						RSICV指令集分为基本指令集I和扩展指令集M,A,F,D,C。基本指令集I是整数指令集,也是RISC-V中,对于任何处理器必须有的指令集,扩展指令集可有可无。
					\item
						基本指令集有六种格式:
					\begin{enumerate}		
						\item 
							R 类型指令:用于寄存器 - 寄存器操作;
						\item 
							I 类型指令:用于短立即数和访存 load 操作;
						\item 
							S 类型指令:用于访存 store 操作;
						\item 
							B 类型指令:用于条件跳转操作;
						\item 
							U 类型指令:用于长立即数操作;
						\item 
							J 类型指令:用于无条件操作;
					\end{enumerate}
					\begin{figure}[!htbp]
						\centering
						\includegraphics[scale=0.8]{img/two.png}
					\end{figure}
					\begin{figure}[!htbp]
						\centering
						\includegraphics[scale=0.5]{img/three.png}
					\end{figure}
				\end{enumerate}

		\subsection{蜂鸟E203简介}
			\subsubsection{E203}
				\begin{enumerate}
					\item 
						蜂鸟 E203 系列处理器由作者所在的公司开发,是一款开源的 RISC-V 处理器。蜂鸟是世
						界上最小的鸟类,其体积虽小,却有着极高的速度与敏锐度,可以说是“能效比”最高的鸟类。
						E203 系列以蜂鸟命名便寓意于此,旨在将其打造成为一款世界上最高能效比的 RISC 处理器。
				\end{enumerate}	
				\begin{figure}[!htbp]
					\centering
					\includegraphics[scale=0.8]{img/four.png}
				\end{figure}
					
			\subsubsection{E203 核心数据通路的模块划分}
				\begin{enumerate}
					\item 
						IFU 取址单元
					\item 
						EXU 执行单元
					\item 
						LSU 访存单元
					\item 
						BIU 总线
				\end{enumerate}	

			\subsubsection{E203 数据通路的两级流程水线}
				\begin{enumerate}
					\item 
						第一级是IFU,包括,取址、分支预测、生成PC。
					\item 
						第二级是译码、派遣、执行、访存、写回。
				\end{enumerate}	

			\subsubsection{E203 的特点}
				\begin{enumerate}
					\item 
						蜂鸟E203 处理器研发团队拥有在国际一流公司多年开发处理器的经验,使用稳健的。
					\item 
						蜂鸟E203 的代码为人工编写,添加丰富的注释且可读性强,非常易于理解。
					\item 
						蜂鸟E203 专为IoT 领域量身定做,其具有2 级流水线深度,功耗和性能指标均优于目前主流商用的ARM Cortex-M 系列处理器,且免费开源,能够在IoT 领域完美替代ARM Cortex-M 处理器。
				\end{enumerate}	

		\subsection{T-core开发板介绍}
			\begin{enumerate}
				\item 
					T-core开发板是友晶科技公司的基于RISC-V的新款开发板。T-Core提供了围绕Intel MAX 10 FPGA构建的强大的硬件设计平台。它配备完善,可在控制平面或数据路径应用中提供具有成本效益的单芯片解决方案,并提供行业领先的可编程逻辑,以实现最终的设计灵活性。
				\item 
					借助MAX 10 FPGA,可以获得比上一代更低的功耗/成本和更高的性能。可支持大量应用,包括协议桥接,电机控制驱动,模数转换和手持设备。T-Core开发板包括硬件,例如板载USB-Blaster II,QSPI Flash,ADC接头连接器,WS2812B RGB LED和2x6 TMD扩展接头连接器。通过利用所有这些功能,T-Core是展示,评估和原型化Intel MAX 10 FPGA真正潜力的理想解决方案。T-Core还通过板载JTAG调试支持RISC-V CPU。它是学习RISC-V CPU设计或嵌入式系统设计的理想平台。
			\end{enumerate}	
	
		\subsection{T-core开发板介绍}
		\begin{enumerate}
			\item T-core开发板是友晶科技公司的基于RISC-V的新款开发板。T-Core提供了围绕Intel MAX 10 FPGA构建的强大的硬件设计平台。它配备完善,可在控制平面或数据路径应用中提供具有成本效益的单芯片解决方案,并提供行业领先的可编程逻辑,以实现最终的设计灵活性。
			\item 借助MAX 10 FPGA,可以获得比上一代更低的功耗/成本和更高的性能。可支持大量应用,包括协议桥接,电机控制驱动,模数转换和手持设备。T-Core开发板包括硬件,例如板载USB-Blaster II,QSPI Flash,ADC接头连接器,WS2812B RGB LED和2x6 TMD扩展接头连接器。通过利用所有这些功能,T-Core是展示,评估和原型化Intel MAX 10 FPGA真正潜力的理想解决方案。T-Core还通过板载JTAG调试支持RISC-V CPU。它是学习RISC-V CPU设计或嵌入式系统设计的理想平台。
		\end{enumerate}	
		\begin{figure}[H]
			\centering
			\includegraphics[scale=1]{img/five.png}
		\end{figure}
		\begin{figure}[H]
			\centering
			\includegraphics[scale=1]{img/six.jpg}
		\end{figure}

	\section{实践篇}
	\subsection{硬件开发}
	
	\subsection{软件开发}
	我们需要根据我们设计的指令格式修改编译器,使其能够编译dot指令,并将其转换为对应的机器码。这样,编译器就可以将包含dot指令的测试程序编译成可执行文件,供硬件执行。
	\begin{enumerate}
		\item 生成操作码宏定义
		\begin{itemize}
			\item 文件包riscv-opcodes中枚举了全部RISC-V指令的操作码信息。因为我们设计的dot指令属于rv32i指令集,因此需要在opcodes-rv32i文件中加入我们编写的dot指令的指令格式。
				\begin{figure}[H]
					\centering  
					\includegraphics[scale=0.3]{img/1.png}   
					\caption{riscv-opcodes文件包}
				\end{figure}
				\begin{figure}[H]
					\centering  
					\includegraphics[scale=0.5]{img/2.png}   
					\caption{修改 opcodes-rv32i 文件}
				\end{figure}
			\item 运用转换脚本parse\_opcodes生成编译器所需要的信息,即C语言格式的宏定义,存放在dot.h中
				\begin{figure}[H]
					\centering  
					\includegraphics[scale=0.8]{img/changeopcodes.JPG} 
					\caption{用脚本转换 opcodes}
				\end{figure}
		\end{itemize}
	
		\item 修改 RISC-V GNU工具链
		\\RISC-V GNU工具链是一组用于支持RISC-V C和C++的交叉编译工具链,这些工具构成了一个完整的系统。GNU工具链包括riscv-gcc、riscv-glibc等子仓库。
		\begin{itemize}
			\item 从gitee上下载完整的工具链后,打开 "riscv-gnu-toolchain/riscv-binutils/include/opcode" 路径下的 riscv-opc.h 文件,将第1步中生成的dot指令相关的定义和代码复制到该文件中。
				\begin{figure}[H]
					\centering  
					\includegraphics[scale=0.8]{img/3.png} 
					\caption{更改工具链}
				\end{figure}
				\begin{figure}[H]
					\centering   
					\includegraphics[scale=0.7]{img/4.png} 
					\caption{更改工具链}
				\end{figure}
			\item 打开 "riscv-gnu-toolchain/riscv-binutils/opcodes" 路径下的 riscv-opc.c 文件,找到定义的 riscv\_opcodes 结构体,在其中添加dot指令
				\begin{figure}[H]
					\centering  
					\includegraphics[scale=0.5]{img/5.png} 
					\caption{更改工具链}
				\end{figure}
			\item 修改完成后,我们需要安装一些编译依赖,然后就可以重新编译生成工具链了,此过程较长,需要20分钟左右。
				\begin{figure}[H]
					\centering  
					\includegraphics[scale=0.8]{img/COMPILE.JPG} 
					\caption{编译工具链}
				\end{figure}
		\end{itemize}
		
		\item 测试指令
			\begin{itemize}
				\item 首先在 dot\_try 文件夹下新建一个 tool\_test
				文件夹,并在其中新建 dot.c 文件,用于测试 dot 指令是否可以被正常编译。
					\begin{figure}[H]
						\centering  
						\includegraphics[scale=0.6]{img/6.png} 
						\caption{编写测试程序}
					\end{figure}
				\item 调用 gcc 编译 dot.c 文件,如果成功修改了工具链,就会生成 dot 文件。
					\begin{figure}[H]
						\centering  
						\includegraphics[scale=0.8]{img/makedot.JPG} 
						\caption{调用 gcc 编译}
					\end{figure}
				\item 进行汇编代码查看。可以在 dot2\_test、dot3\_test、dot4\_test 函数的片段中,找到生成的 dot 汇编指令。
					\begin{figure}[H]
						\centering  
						\includegraphics[scale=0.7]{img/7.png} 
						\caption{查看 dot 指令}
					\end{figure}		
			\end{itemize}
		\item 创建程序文件
		\begin{itemize}
			\item 在 "demo\_dot" 文件夹下创建一个 "demo\_dot.c" 的文本文档。包含以下头文件。
			\begin{lstlisting}[language={C++}]
				#include <stdio.h>
				#include <stdlib.h>
				#include <stdbool.h>
				#include <stdatomic.h>
				#include "encoding.h"
				#include <platform.h>  
			\end{lstlisting}
			\item 宏定义,三个常数 CONST\_I、CONST\_K、CONST\_J 分别为 32、64、32,定义标识符
			A\_ROW(矩阵 A 的行)为 CONST\_I,定义 A\_COL(矩阵 A 的列)为 CONST\_K;定义标识符
			B\_ROW(矩阵 B 的行)为 CONST\_K,定义 B\_COL(矩阵 B 的列)为 CONST\_J;定义标识符
			RES\_ROW(乘法运算得到的新矩阵的行)为 CONST\_I,定义 A\_COL(乘法运算得到的新矩阵的
			列)为 CONST\_J。
			\begin{lstlisting}[language={C++}]
				// matrix dimensions
				// CONST_K must set as multiple of 4 for dot instruction
				#define CONST_I 32
				#define CONST_K 64
				#define CONST_J 32
				//a[const_i][const_k]
				#define A_ROW CONST_I
				#define A_COL CONST_K
				//b[const_k][const_j]
				#define B_ROW CONST_K
				#define B_COL CONST_J
				//res[const_i][const_j]
				#define RES_ROW CONST_I
				#define RES_COL CONST_J  
			\end{lstlisting}
			\item 定义 demo\_uart\_init 函数,用于初始化,首先配置 GPIO\_IOF\_EN 对应的比特位为 1,使能 IOF 模
			式,再将 GPIO\_IOF\_SEL 对应的比特位清零,选择 IOF0,再配置 UART\_DIV 寄存器,设置波特
			率约为 115200。最后配置 UART\_TXCTRL 和UART\_RXCTRL,将 UART0 的 TX 和 RX 使能。
			\begin{lstlisting}[language={C++}]
				void uart_init(){
					// Configure UART to print
					GPIO_REG(GPIO_IOF_EN) |= IOF0_UART0_MASK;
					GPIO_REG(GPIO_IOF_SEL) &= ~IOF0_UART0_MASK;
					// 115200 Baud Rate
					// get_cpu_freq() / baud_rate - 1, and get_cpu_freq() = 16MHz
					UART0_REG(UART_REG_DIV) = 138;
					UART0_REG(UART_REG_TXCTRL) |= UART_TXEN; // enable tx
					UART0_REG(UART_REG_RXCTRL) |= UART_RXEN; // enable rx
				}
			\end{lstlisting}
			\item 主函数中,首先进行 UART 的初始化, 并定义 i、j、k 变量分别用于遍历矩阵的行和列,reg 变量作为 dot 指令算法输出矩阵的寄存器,incr变量作为运算步长。
			\begin{lstlisting}[language={C++}]
				int main(int argc, char *argv[]) {
					uart_init();
					int i = 0, j = 0, 1 k = 0, reg = 0, incr = 4;			
				\end{lstlisting}
				\item 打印 "Malloc and initial Matrixs!!" 字符串,定义四个指针变量,并使用 malloc 函数分配 A 矩阵、B矩阵、未使用 dot 指令进行矩阵相乘得到的新矩阵、使用 dot 指令进行矩阵相乘得到的新矩阵所需的内存空间,并返回一个指针,指向已分配大小的内存。
				\begin{lstlisting}[language={C++}]
					// malloc and init matrixs
					printf("Malloc and initial Matrixs!!\n\n");
					// matrix stored as an array of size row * column
					int *a = NULL, *b = NULL, *no_dot_res = NULL, *dot_res = NULL;
					a = (int*) malloc(A_ROW * A_COL * sizeof(int));
					b = (int*) malloc(B_ROW * B_COL * sizeof(int));
					no_dot_res = (int*) malloc(RES_ROW * RES_COL * sizeof(int));
					dot_res = (int*) malloc(RES_ROW * RES_COL * sizeof(int));			
				\end{lstlisting}
				\item 初始化矩阵 A 和矩阵 B,初始化矩阵 A 和矩阵 B 相乘的两种运算方式的结果矩阵。
				\begin{lstlisting}[language={C++}]
					// initialization matrix A
					for(i = 0; i < A_ROW; i++) {
						for (j = 0; j < A_COL; j++) {
							a[i * A_COL + j] = i * A_COL + j;
						}
					}
					// initialization matrix B
					for(i = 0; i < B_ROW; i++) {
						for (j = 0; j < B_COL; j++) {
							b[i * B_COL + j] = i * B_COL + j;
						}
					}
					// initialization matrix no_dot_res,dot_res
					for(i = 0; i < RES_ROW; i++) {
						for (j = 0; j < RES_COL; j++) {
							no_dot_res[i * RES_COL + j] = 0;
							dot_res [i * RES_COL + j] = 0;
						}
					}		
				\end{lstlisting}
				\item RISC-V 定义了 3 个 64 位计数器,分别为:instret、cycle、time,这三个寄存器可以用来评估硬件性能。其中,instret 计数器统计自 CPU 复位以来共运行了多少条指令;cycle 计数器统计自 CPU 复位以来共运行了多少个周期;time 计数器统计自CPU 复位以来共运行了多少时间,驱动 time 计数器是已知的固定频率的时钟,例如 32768Hz 的时钟。
				采用常规算法进行矩阵的乘法运算。并调用 get\_instret\_value()、get\_cycle\_value()、
				get\_timer\_value() 三个函数,通过计算得到这种运算方式的指令数、周期数以及运行的时间。
				\begin{lstlisting}[language={C++}]
					// no dot instruction(traditional tile matrix multipication)
					printf("Matrix multiplication without using custom DOT instruction:
					\n");
					unsigned int no_dot_instret_start = get_instret_value();
					unsigned int no_dot_cycle_start = get_cycle_value();
					unsigned int no_dot_timer_start = get_timer_value();
					for (i = 0; i < RES_ROW; i++) {
						for (j = 0; j < RES_COL; j++) {
							for (k = 0; k < CONST_K; k++) {
								no_dot_res[i * RES_COL + j] += a[i * A_COL + k] * b[k *
								B_COL + j];
							}
						}
					}
					unsigned int no_dot_timer_cost = get_timer_value() -
					no_dot_timer_start;
					unsigned int no_dot_cycle_cost = get_cycle_value() -
					no_dot_cycle_start;
					unsigned int no_dot_instret_cost = get_instret_value() -
					no_dot_instret_start;
					printf("not_dot time cost: %0.2fms\n",
					(float)no_dot_timer_cost/RTC_FREQ*1000);
					printf("not_dot_cycle: %u\n", no_dot_cycle_cost);
					printf("not_dot_instret: %u\n", no_dot_instret_cost);
					printf("not_dot CPI: %.2f\n\n",
					(float)no_dot_cycle_cost/no_dot_instret_cost);	
				\end{lstlisting}
				\item 采用 dot2、dot3、dot4 指令进行矩阵的乘法运算。并调用 get\_instret\_value()、get\_cycle\_value()、get\_timer\_value()三个函数,通过计算得到这种运算方式的指令数、周期数以及运行的时间。
				\begin{lstlisting}[language={C++}]
					unsigned int dot_instret_start = get_instret_value();
					unsigned int dot_cycle_start = get_cycle_value();
					unsigned int dot_timer_start = get_timer_value(); 
					
					for (i = 0; i < RES_ROW; i++) {
						for (j = 0; j < RES_COL; j++) {
							int k = 0;
							
							asm volatile (
							"lw x10, %[a0]\t\n"
							"lw x11, %[a1]\t\n"
							"lw x12, %[a2]\t\n"
							"lw x13, %[a3]\t\n"
							"lw x14, %[a4]\t\n"
							"lw x15, %[a5]\t\n"
							"lw x16, %[a6]\t\n"
							"lw x17, %[a7]\t\n"
							"dot4 %[output], x12, x13\t\n"
							: [output]"=r"(reg)
							: [a0]"m"(a[i * A_COL + (k + 0)])
							,[a1]"m"(a[i * A_COL + (k + 1)])
							,[a2]"m"(a[i * A_COL + (k + 2)])
							,[a3]"m"(a[i * A_COL + (k + 3)])
							,[a4]"m"(b[(k + 0) * B_COL + j])
							,[a5]"m"(b[(k + 1) * B_COL + j])
							,[a6]"m"(b[(k + 2) * B_COL + j])
							,[a7]"m"(b[(k + 3) * B_COL + j])
							: "x10","x11","x12","x13"
							,"x14","x15","x16","x17"
							);
							
							dot_res[i * RES_COL + j] += reg;
							k += 4;
							
							asm volatile (
							"lw x10, %[a0]\t\n"
							"lw x11, %[a1]\t\n"
							"lw x12, %[a2]\t\n"
							"lw x13, %[a3]\t\n"
							"lw x14, %[a4]\t\n"
							"lw x15, %[a5]\t\n"
							"dot3 %[output], x12, x13\t\n"
							: [output]"=r"(reg)
							: [a0]"m"(a[i * A_COL + (k + 0)])
							,[a1]"m"(a[i * A_COL + (k + 1)])
							,[a2]"m"(a[i * A_COL + (k + 2)])
							,[a3]"m"(b[(k + 0) * B_COL + j])
							,[a4]"m"(b[(k + 1) * B_COL + j])
							,[a5]"m"(b[(k + 2) * B_COL + j])
							: "x10","x11","x12","x13"
							,"x14","x15"
							);
							
							dot_res[i * RES_COL + j] += reg;			   
						}
					}
					
					unsigned int dot_timer_cost   = get_timer_value() - dot_timer_start;
					unsigned int dot_cycle_cost   = get_cycle_value() - dot_cycle_start;
					unsigned int dot_instret_cost = get_instret_value() - dot_instret_start;
					
					printf("dot time cost: %.2fms\n", (float)dot_timer_cost/RTC_FREQ*1000);
					printf("dot_cycle: %u\n", dot_cycle_cost);
					printf("dot_instret: %u\n", dot_instret_cost);
					printf("dot CPI: %.2f\n\n", (float)dot_cycle_cost/dot_instret_cost);	
				\end{lstlisting}
				\item 对比两种运算方式的结果是否一致,来验证采用 dot 指令进行矩阵的乘法运算的结果是否是正确的。
				\begin{lstlisting}[language={C++}]
					// verify the no_dot_res and dot_res array are equal
					printf("Matrix multiplication result verification: \n");
					int verifyRes = 1;
					for(i = 0;i < RES_ROW * RES_COL; i++) {
						if(dot_res[i] != no_dot_res[i]) {
							verifyRes = 0;
							break;
						}
					}
					if(verifyRes)
					printf("Pass!\n\n");
					else
					printf("Fail!\n\n");	
				\end{lstlisting}
				\item 计算两种运算方式的指令数、周期数以及运行的时间的比值,并打印出来。最后释放由 malloc() 函数申请的内存空间。
				\begin{lstlisting}[language={C++}]
					printf("Ratio of timer: %.2f\n",
					(float)no_dot_timer_cost/dot_timer_cost );
					printf("Ratio of cycle: %.2f\n",
					(float)no_dot_cycle_cost/dot_cycle_cost );
					printf("Ratio of instret (retired instruction): %.2f\n",
					(float)no_dot_instret_cost/dot_instret_cost );	
					// free matrix
					free(a);
					free(b);
					free(no_dot_res);
					free(dot_res);
					return 0;
				}
			\end{lstlisting}
			
		\end{itemize}
	\end{enumerate}	

	\section{结果展示}	
	
	
	\section{未来展望}
	
	
\end{document}